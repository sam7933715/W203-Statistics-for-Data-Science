%%%%%%%%%%%%%%%%%%%%%%%%%%%%%%%%%%%%%%%%%%%%%%%%%%%%%%%%%%%%%%%%%%%%%%%%%%%%
% Statistics for Data Science (DataSci w203)
% Week 4 homework
%%%%%%%%%%%%%%%%%%%%%%%%%%%%%%%%%%%%%%%%%%%%%%%%%%%%%%%%%%%%%%%%%%%%%%%%%%%%

% Setup
\documentclass[12pt,a4paper]{article}
\usepackage[inner=1.5cm,outer=1.5cm,top=2.5cm,bottom=2.5cm]{geometry}
\usepackage{graphicx}
\usepackage[english]{babel}
\usepackage{amsmath}
\usepackage{amssymb}
\numberwithin{equation}{subsection}
\usepackage{hyperref}

%%%%%%%%%%%%%%%%%%%%%%%%%%%%%%%%%%%%%%%%%%%%%%%%%%%%%%%%%%%%%%%%%%%%%%%%%%%%


\begin{document}

\title{Statistics for Data Science \\
       Unit 4 Homework: Random Variables}
%\institute{Institution}
\maketitle

%----------------------------------------------------------------------------------------
%----------------------------------------------------------------------------------------
\begin{enumerate}

\item \textbf{Best Game in the Casino}

You flip a fair coin 3 times, and get a different amount of money depending on how many heads you get. For 0 heads, you get \$0. For 1 head, you get \$2. For 2 heads, you get \$4. Your expected winnings from the game are \$6. 

\begin{enumerate}
\item How much do you get paid if the coin comes up heads 3 times?
\item Write down a complete expression for the cumulative probability function for your winnings from the game.
\end{enumerate}

\item \textbf{Processing Pasta}

A certain manufacturing process creates pieces of pasta that vary by length.  Suppose that the length of a particular piece, $L$, is a continuous random variable with the following probability density function.

$$f(l) = \begin{cases} 0, &l \leq 0 \\
l/2, &0 < l \leq 2 \\ 
0, &2 < l
\end{cases}
$$

\begin{enumerate}
\item Write down a complete expression for the cumulative probability function of $L$.
\item Using the definition of expectation for a continuous random variable, compute the expected length of the pasta, $E(L)$.
\end{enumerate}

\item \textbf{The Warranty is Worth It}

Suppose the life span of a particular (shoddy) server is a continuous random variable, T, with a uniform probability distribution between 0 and 1 year.  The server comes with a contract that guarantees you money if the server lasts less than 1 year.  In particular, if the server lasts $t$ years, the manufacturer will pay you $g(t)= \$100(1-t)^{1/2}$.  Let $X = g(T)$ be the random variable representing the payout from the contract.

\begin{enumerate}
\item Compute the expected payout from the contract, $E(X) = E(g(T))$, using the expression for the expectation of a function of a random variable.
\item Next, compute $E(X)$ another way, by first characterizing the random variable X.  Follow these steps:
\begin{enumerate}
\item First, suppose that you are given a specific value for the payoff, $X = x$, where $ \$0 \leq x \leq \$100 $.  What is the value for T that results in this payoff?
\item Next, suppose that all you know is that the payoff is less than or equal to some value, $X \leq x$, where again $ \$0 \leq x \leq \$100 $.  What does this tell you about the life span of the server?  Specifically, write down the condition for T that is equivalent to $X \leq x$.
\item Using the condition you just wrote down, what is the probability that $X \leq x$?  Write down a complete expression for the cumulative probability function of X.
\item Take a derivative to compute the probability density function for X.
\item Use the pdf of X to compute $E(X)$.  If you did everything right, your answer should match what you got in part (a).
\end{enumerate}
\end{enumerate}

\item \textbf{The Baseline for Measuring Deviations}

Given any random variable $X$ and a real number $t$, we can define another random variable $Y = (X - t)^2$. In other words, for any random variable $X$, we can choose a real number, $t$, as a baseline and calculate the squared deviation of $X$ away from $t$.

You might wonder why we often square deviations (instead of taking an absolute value, or cubing them, etc.).  This exercise will shed some light on why this is a natural choice.

\begin{enumerate}
\item Write down an expression for $E(Y)$ and use properties of expectation to simplify it as much as you can.
\item Taking a partial derivative with respect to $t$, compute the value of $t$ that minimizes $E(Y)$.  (Hint: Your answer should be a very familiar value)
\item What is the value of $E(Y)$ for this choice of $t$?
\end{enumerate}

\item \textbf{Optional Advanced Exercise: Characterizing a Function of a Random Variable}

Let $X$ be a continuous random variable with probability density function $f(x)$, and let $h$ be an invertible function where $h^{-1}$ is differentiable.  Recall that $Y = h(X)$ is itself a continuous random variable.  Prove that the probability density function of $Y$ is 

$$g(y) =f(h^{-1}(y)) \cdot \left| \frac{d}{dy}h^{-1}(y) \right| $$

\end{enumerate}

\end{document}